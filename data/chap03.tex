
\chapter{系统设计}
\label{chap:design}

前面两章介绍了本系统的研究目的,要解决的问题和对于这些问题本系统提出的解决方案。虽然第二章中只简单提到了解决方案,而并没有涉及系统具体的逻辑结构。但是从第二章的介绍,也不难看出本文介绍的系统大概的系统架构与主要的功能设置。接下来的两章我们将深入的介绍该系统的逻辑结构和部分实现细节。下面将通过“统一接口”、“文档撰写与展示”、“用户空间逻辑结构”、“社交网络与门户”、“文档认证协作”、“外围辅助程序”几个部分,逐步介绍本系统的逻辑结构。

\section{统一接口}
\label{sec:restful}

前文提到,本系统要实现的一个重要功能就是使用户文档可以方便的在不同种类的设备间传递,这些设备包括运行windows操作系统的PC机、运行mac操作系统的apple机、运行ios平台和android平台手机和平板电脑、运行linux操作系统的实验用机和各种移动阅读设备(电纸书)等等。显然,要实现这样的功能,就要求任何设备上的客户端程序都可以访问系统服务器上的数据,并且访问的方式要相对统一。这样才能尽可能的降低服务器端程序的复杂度,使系统结构尽量紧凑。

面向统一接口的web service服务架构,显然是实现如上功能的不二选择。web service的主要特点就是:客户端访问web service只需要通过因特网标准协议,如HTTP或XML。因为HTTP协议和XML都是与平台无关的标准协议,因此,可以被任何主流操作系统正确理解和解释。

web service的常用的方法有:
\begin{enumerate}
\item RPC 所谓的远程过程调用 (面向方法):像调用本地服务(方法)一样调用服务器的服务(方法),通常的实现有XML-RPC,JSON-RPC,通信方式基本相同, 所不同的只是传输数据的格式。
\item SOA 所谓的面向服务的架构(面向消息):前几年炒的很火的一个词, SOA是基于消息的,通常与具体的实现语言无关, 所以在一定程度上得到大公司的支持。
\item REST 所谓的 Representational state transfer (面向资源):是以资源为中心, 名词即资源的地址, 动词即施加于名词上的一些有限操作, 表达是对各种资源形态的抽象。
\end{enumerate}
本系统选择架构比较清晰,可扩展性较好,也是目前被广泛提倡使用的REST结构。REST 是英文 Representational State Transfer 的缩写,是近年来迅速兴起的,一种基于 HTTP,URI,以及 XML 这些现有协议与标准的,针对网络应用的设计和开发方式。它可以降低开发的复杂度,提高系统的可伸缩性。REST 的核心是可编辑的资源及其集合,用符合 Atom 文档标准的 Feed 和 Entry 表示。每个资源或者集合有一个惟一的 URI。系统以资源为中心,构建并提供一系列的 Web 服务。REST 的基本概念和原则包括:系统上的所有事物都被抽象为资源、每个资源对应唯一的资源标识、对资源的操作不会改变资源标识本身、所有的操作都是无状态的等等。

在 REST 中,开发人员显式地使用 HTTP 方法,对系统资源进行创建、读取、更新和删除的操作:
\begin{enumerate}
\item 使用 POST 方法在服务器上创建资源
\item 使用 GET 方法从服务器检索某个资源或者资源集合
\item 使用 PUT 方法对服务器的现有资源进行更新
\item 使用 DELETE 方法删除服务器的某个资源
\end{enumerate}
\begin{figure}[H]
  \centering
  \includegraphics{restful}
  \caption{系统提供web端和restful api接口结构图}
  \label{fig:xfig7}
\end{figure}
