
\chapter{解决方案}
\label{chap:contribute}

在这个网络如此发达,web2.0和云计算的使用日益普及的时代,人们的文档管理方式也发生了巨大的改变。但是由于种种原因,有些好的模式和方法并没有被高校用户所认可与广泛使用。而且可以证实,有些公共网络上的文档服务也并不适合高校用户。所以就导致了大部分高校用户的个人文档管理方式依然停留在10年前的单机模式下,也就势必带来本文上一章所提到的诸多文档管理的问题。而本文描述系统的设计目标就是帮助高校用户解决这些问题。本章下面就针对上面提到的每一个问题逐一提出本系统的解决方案。

\section{文档写作解决方案}
\label{sec:write}

\subsection{LaTex}
\label{sec:latex}

上一章提到,高校用户文档写作过于依赖Word软件,带来的诸多弊端。但是,在很多人看来Word依然是用来写文档的绝佳选择。在我国,Word的普及率如此之高,它有的时候甚至会被认为是编辑文档的唯一选择。仅“所见即所得”一项,Word就会赢得绝大多数用户的心。但是很多人也许不知道,在国外,对于写学术报告和科技论文,有更加标准也更加普及的编辑工具,那就是\LaTeX,世界上很多著名的出版机构斗接受或强制要求作者使用\LaTeX稿件,接受 LaTeX 稿件的出版社大都有自己的文稿样式模板,主要就是一个类型文件包,简称类包。如果稿件未被甲出版社采用,在转投乙出版社前,只需将稿件第一句中类包名称由甲出版社的改为乙出版社的,整篇稿件的样式就随之自动转换过来了,确实很方便。国外知名大学也大多都要求学生使用\LaTeX编写科技报告和论文,这和国内Word一家独大的局面截然不同。本系统用来解决文档编辑问题的基础之一就是\LaTeX。那么什么是\LaTeX呢?

\LaTeX是一种排版系统,它基于\TeX\footnote{国计算机教授高德纳在1978年编写的功能强大的排版软件,高德纳最早开始自行编写它的原因是当时十分粗糙的排版水平已经影响到他的巨著《计算机程序设计艺术》(The Art of Computer Programming)的印刷质量。他以典型的黑客思维模式,最终决定自行编写一个排版软件。}排版系统并由此发展而来。它是由美国电脑学家莱斯利·兰伯特在20世纪80年代初期开发,利用这种格式,即使用户没有排版和程序设计的知识也可以充分发挥由\TeX所提供的强大功能,能在几天,甚至几小时内生成很多具有书籍品质的印刷品。对于生成复杂表格和数学公式,这一点表现得尤为突出。因此它非常适用于生成高印刷质量的科技和数学类文档。这个系统同样适用于生成从简单的信件到完整书籍的所有其他种类的文档。与其他的文字排版系统(比如Word)相比,\LaTeX最突出的优势就是高质量、高专业水准的文稿排版效果。下面简单比较一下\LaTeX和Word:
\begin{description}
\item[入门难度]  Word 特点就是“所见即所得”,其基本功能初学者很容易掌握,很多 Word 用户都是无师自通。但随着篇幅和复杂程度的增加,花费在文稿格式上的精力和时间要明显加大,如图~\ref{fig:xfig2}蓝色示意曲线所示。因为创建自定义编号、交叉引用、索引和参考文献等就不是“所见即所得”了,得耐着性子反复查阅 Word 的在线帮助或借助相关软件帮忙。
对于 \LaTeX 初学者,即就是编排很简单的文章,也要花较多的精力和时间去学习那些枯燥的命令和语法,特别是排写数学公式,经常出错,多次编译不能通过,使很多初学者望而却步。可是一旦掌握,不论文 稿长短和复杂与否都会熟练迅速地完成,先前学习 \LaTeX 的精力投入将由此得到回报,如图~\ref{fig:xfig2}红色示意曲线所示。关于\LaTeX入门难度大的问题,本系统下面介绍\smarkdown的时候会给出解决方案。
\begin{figure}[H]
  \centering
  \includegraphics{latexVSword}
  \caption{latex与word的学习曲线比较}
  \label{fig:xfig2}
\end{figure}
\item[内容与格式] 当用 Word 写作时,要花很多精力对页版式、章节样式、字体属性、对齐和行距等文本参数进行反复选择对比,尤其是长篇文章,经常出现因疏忽而前后文体格式不一致的现象;当在稿件中插入或删除一章或章节次序调整时,各章节标题、图表和公式等的编号都要用手工作相应修改,稍有不慎就会出现重号或跳号。 我们既是作者又是编辑还兼排字工。

使用 \LaTeX 编版,如无特殊要求,只要将文稿的类型(article、report 或 book 等)告诉 \LaTeX,就可专心致志地写文章了,至于文稿样式的各种细节都由 \LaTeX 统一规划设置,而且非常周到细致;当修改稿件时,其中的章节、图表和公式等的位置都可任意调整,无须考虑编号,因为在源文件里就没有编号,文件中的所有编号都是在最后编译时 \LaTeX 自动统一添加的,所以绝对不会出错。
换句话说,Word 把文稿的内容与样式混为一体,而 \LaTeX 将它们分离,作者只需专注于文稿的内容,而文稿的样式几乎不用过问,\LaTeX 是你的聪明而忠诚的文字秘书,如有特殊要求,也可使用命令修改,\LaTeX 会自动将相关设置更新,无一遗漏。
\item[数学公式问题] 

Word 有个公式编辑器,可以编辑普通数学公式,但使用很不方便,外观效果较差,也不能自动编号,尤其是很难作为文本的一部分,融入某一行中,大都专起一行。如果碰到复杂的数学公式,编辑起来就很困难。有些用户只好另外安装可嵌入 Word 环境的工具软件 Math-Type 来弥补这一不足。

\LaTeX 的特长之一就是数学公式编辑,方法简单直观,“所想即所得”,公式的外观精致细腻,而且公式越复杂这一优点就越明显。普通单行公式可以像纯文字文本一样插入字里行间。下面举三个例子比较一下 ,其中图~\ref{fig:xfig3}是 DOC 格式的屏幕显示效果,图~\ref{fig:xfig4}是将 DOC 格式文件通过 Acrobat 转换为 PDF 格式的效果,图~\ref{fig:xfig5}是\LaTeX生成的PDF格式的效果,显然\LaTeX的数学公式排版效果更好。
\begin{figure}[H]
  \centering
  \includegraphics{wordformula}
  \caption{Word显示数学公式效果}
  \label{fig:xfig3}
\end{figure}
\begin{figure}[H]
  \centering
  \includegraphics{wordformulapdf}
  \caption{Word转成pdf后显示数学公式效果}
  \label{fig:xfig4}
\end{figure}
\begin{figure}[H]
  \centering
  \includegraphics{latexformula}
  \caption{LaTeX里显示数学公式}
  \label{fig:xfig5}
\end{figure}
\item[参考文献] Word 目前还不具备管理参考文献的功能,用户一般都是采用 Reference Manager 或是 NoteExpress 等外部工具软件来解决这一问题。
而创建参考文献却是 \LaTeX 的强项。\LaTeX 自带一个辅助程序 BibTeX,它可以根据作者的检索要求,搜索一个或多个文献数据库,然后自动为文稿创建所需的参考文献条目列表。如果编写其它文件用到相同的参考文献时可直接引用这个数据库。参考文献的样式和排序方式都可以自行设定。很多著名的科技刊物出版社、学术组织和 TUG 网站等都提供相关的 BibTeX 文献数据库文件,可免费下载。
\item[稳定性和安全性] 一篇科技论文少则几十页,多则上百页,其中含有许多图形和公式(Word 将公式处理为图形),正是由于 Word“所见即所得”,论文中的图形都要完整地插入页面 。随着文件的篇幅增大图形数量增多,处理速度明显减慢。编写一篇论文要无数次地打开、保存和关闭,往往要长时间等待甚至死机或文稿无法打开,所以 Word 经常出现“文件恢复”提示信息,但其中的图形很有可能丢失,取而代之的是一个小红叉。如果将文件分解为多个子文件,可以缓解这一问题,但又会出现难以自动创建目录、索引和参考文献等新问题;若章节、图表和公式需要 在子文件之间调换调整,那编号就全乱套了。
\LaTeX 是纯文本文件,所有图形都是在最后编译时调入。同一篇文章,其 \LaTeX 源文件只有 Word 文件尺寸的几十分之一。所以,\LaTeX 源文件的长短,不会对文件存取和编辑过程产生明显影响。\LaTeX 也允许采用多个子文件,章节和图表可随意增删,\LaTeX 是在最后编译时才将所有子文件汇总排序,生成统一的文件页码、标题序号、图表和公式编号以及各种目录。
Word 从问世到现在不断地更新版本,并经常要求下载补丁程序,防止病毒攻击。\LaTeX 及其前身 \TeX,近二十年来,没有发现系统漏洞,即使有,造成源文件损坏的风险也是微乎其微;迄今也未发现任何宏包含有病毒。
\item[通用性] 随着计算机软硬件性能的提高,在 PC 机上使用 Unix/Linux、Mac OS 或其他操作系统的用户越来越多。由于 \LaTeX 系统的程序源代码是公开的,因此人们开发了用于各种操作系统的版本,而且 \LaTeX 源文件全部采用国际通行的 ASCII 字符,所以 LaTeX 及其源文件可以毫无阻碍地跨平台 、跨系统使用和传播。而 Word 只能在 Windows 操作系统上运行。
\end{description}
综上所述,比起 Word , \LaTeX在编写科技报告和科技论文的领域趋势有着很多先天优势。但是由于\LaTeX的学习曲线比起 Word 确实比较陡峭,因为\LaTeX需要在文本中添加的标签比较多,这本身就破坏来文档源文件的可读性,同样使作者不能专心于内容。图~\ref{fig:xfig6}是本文制作作者在编写本文时的源文件截图,编辑器为emacs\footnote{由Richard Stallman于1975年在MIT协同盖伊·史提尔二世共同完成的宏编辑器}。显然这并不是很符合我们写作的习惯,标签在文档中对于内容的干扰还是比较明显的,这多少会减弱作者对内容的专注。所以只有\LaTeX并不是我们的最终解决方案。它不足的地方我们的系哦用下面提到的\smarkdown来补充。
\begin{figure}[H]
  \centering
  \includegraphics{latexsource}
  \caption{LaTeX源文件}
  \label{fig:xfig6}
\end{figure}





\section{文档存储解决方案}
\label{sec:save}

\section{文档分发解决方案}
\label{sec:giveout}

\section{个人成果展示方案}
\label{sec:display}

\section{项目管理解决方案}
\label{sec:project}




