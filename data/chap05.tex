
\chapter{总结和展望}
\label{chap:sumandprospect}

由于本系统设计理念相对新颖,结构也较其他信息管理系统复杂,而且涉及的技术繁多,如果一次完成如此庞大的系统难度较大。所以,本系统在实现步骤上将分成三期:
\begin{description}
\item[第一期] 实现系统的总体架构,并使用敏捷开发的方法逐步迭代,完成系统主要的webservice服务器端设计和编码。提供基本完善的RESTful api,并实现相对功能完善的WEB端用户界面和基本的版本控制能力。建立起以图书馆为主要服务提供者的基本信息和认证服务。并通过多种途径在学生和老师中积极宣传。争取在本校积累起一定的用户量。实现本校用户的联合认证登陆,并实现与论文提交,成果认证,查收查引等校园服务无缝整合,以实现本校师生信息化文档管理。
\item[第二期] 实现官方的windows,linux, mac和移动客户端程序。实现系统的本地客户端和WEB界面同时访问,真正实现系统的多设备无障碍访问。并在横向和纵向上积极推广系统,一方面增加系统用户量,并把用户群从本校推广到其他高校和科研机构。另外一方面积极配合学校其他信息系统,实现本系统与其他系统的深度整合,进一步为师生的教学和科研提供信息化的便捷服务。积极搞好本系统的初级和高级操作的培训和基于统一接口的开发培训。鼓励更多的师生参与到本系统的开发中来。此外,系统也将加入更加完善的版本控制。
\item[第三期] 实现了一,二两期后,本系统可以说基本上实现了要实现的目标,完成了设计中所有的功能,也基本解决了高校用户在文档管理中遇到的诸多问题。此时,系统应该拥有了一定数量的用户。并积累了一定量的文档。那么针对这些用户和文档进行数据分析,将是第三期的主要工作。通过对大量的数据和应用很好的分析算法,可以获得很多有趣的分析结果,比如某些学科的老师感兴趣的文章有哪些,某学校的某位老师是否本领域有很到的关注度等等,这些信息对于学校对人才的评定和学科的热门程度的判断都具有非凡的意义。而且通过数据分析,系统也可以实现更加智能的功能(系统推荐感兴趣的文档等)。除此以外,系统还将增加对R语言\footnote{一门热门的数据统计语言,在国外,被认为是高校学生必学的工具语言}的支持。也将增加项目管理(类Basecamp)和 GTD(个人事务管理)等功能,方便用户管理日常工作和生活。
\end{description}

可以看到,系统的第一期可以说是''雏形'',第二期可以说是''成型'',那么第三期完成后才真正可以说是''完型''。

以上简单的描述了本系统计划实现的三个阶段,截止本论文截稿为止,系统的第一期建设已经进入敏捷开发的迭代编码阶段,很快将在本校内发布。系统的第二期建设将在第一期发布后马上进入编码。对于系统的培训和文档也在积极准备中。

当然系统要真正实现理想中的状态,不仅需要技术上完美的实现,有时也需要积极的推广,学校各职能部门大力支持,和广大师生的积极相应。但是有一点本人始终坚信: 只要系统的功能完善,界面友好,运行稳定,简单易用。那么,无论是刚刚走进大学校园的新生还是在教学科研第一线工作多年的专家教授,都会逐渐习惯并喜欢本系统为文档管理模式带来的改变,并成为本系统忠实的用户。
