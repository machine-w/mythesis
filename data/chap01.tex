
\chapter{概述}
\label{chap:intro}

本系统的设计与实现是天津大学图书馆建立学术科研和教育教学提供支撑系统体
系的一部分。原本要设计的是一个高校机构库管理系统。主要功能是收集并整理
本校老师和博硕士的科研成果,并组织分类后通过统一的发布平台进行展示。要
实现的功能对本校科研水平展示和对本校学术资源的收集。但是通过对厦门大学
和中科院等国内知名机构库系统的调研后发现:
\begin{enumerate}
\item 老式机构库系统从访问量上并没有达到预想的效果。由于现在更多的人习惯使
  用搜索引擎,博客等渠道作为获取信息的主要途径。所以目前国内主要的高校
  机构库平台访问量都不理想。基本上属于 “信息孤岛” 状态。
\item 老式机构库从内容的管理比较繁琐,信息实时性差,更新费时费力。由于
  老式机构库从结构上依然属于信息资源管理系统(erp),其主要的建库和信
  息维护工作还是需要由,机构库的管理部门(比如图书馆) 来完成,而这些部
  门需要维护的信息来源却是学校里的教授或者学生。这就出现了信息多次传递
  的问题。带来的后果自然是管理部门费时费力,教授和学生不满意。
\item 老式机构库从系统实现上国内还是以dspace,Eprints,超星等平台为主,
  其内容的实现机理也是基本上面向数据存储的。这就给信息的结构化扩展带来
  了很多问题。
\end{enumerate}
除了以上列出的几个方面的问题以外,老式机构库追究其问题的根源是:这些系
统是面向信息收集和信息管理部门而设计的。其更多的考虑了信息如何收集和展
示。而忽略了信息来源(即老师和学生)在系统中的作用。管理部门更像实在自
说自话,不断的向一个很少有人访问的门户网站上添加内容。而更加悲剧的是甚
至这些信息的创作者有时都不知道还有这么一个平台上有他们写的东西。这种有
着强烈web 1.0味道的应用在当今这个时代
是必然会被淘汰的。\\