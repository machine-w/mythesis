
\chapter{概述}
\label{chap:intro}

\section{设计理念}
\label{sec:dongji}

本系统的设计与实现是天津大学图书馆建立学术科研和教育教学提供支撑系统体系的一部分。原本要设计的是一个高校机构库管理系统。主要功能是收集并整理本校老师和博硕士的科研成果,并组织分类后通过统一的发布平台进行展示。要实现的功能对本校科研水平展示和对本校学术资源的收集。但是通过对厦门大学和中科院等国内知名机构库系统的调研后发现:
\begin{enumerate}
\item 老式机构库系统从访问量上并没有达到预想的效果。由于现在更多的人习惯使用搜索引擎,博客等渠道作为获取信息的主要途径。所以目前国内主要的高校机构库平台访问量都不理想。基本上属于 “信息孤岛” 状态。
\item 老式机构库从内容的管理比较繁琐,信息实时性差,更新费时费力。由于老式机构库从结构上依然属于信息资源管理系统(erp),由于管理类系统的操作界面相对复杂,加之信息录入工作比较繁琐,所以大多老师学生不愿意使用。所以其主要的建库和信息维护工作还是需要由,机构库的管理部门(比如图书馆)来完成,而这些部门需要维护的信息来源却是学校里的教授或者学生。这就出现了信息多次传递的问题。带来的后果自然是管理部门费时费力,教授和学生不满意。
\item 老式机构库从系统实现上国内还是以dspace,Eprints,超星等平台为主,其内容的实现机理也是基本上面向数据存储的。这就给信息的结构化扩展带来了很多问题。
\end{enumerate}
除了以上列出的几个方面的问题以外,老式机构库追究其问题的根源是:这些系统是面向信息收集和信息管理部门而设计的。其更多的考虑了信息如何收集和展示。而忽略了信息来源(即老师和学生)在系统中的作用。管理部门更像实在自说自话,不断的向一个很少有人访问的门户网站上添加内容。而更加悲剧的是甚至这些信息的创作者有时都不知道还有这么一个平台上有他们写的东西。这种有着强烈web 1.0味道的应用在当今这个时代是必然会被淘汰的。

由于考虑到以上诸多问题,所以本系统开发之初即转换了设计理念。放弃了机构库面向信息收集、管理、发布为主要目标的设计,采用了面向信息创作者,为信息创作者提供便捷易用的个人文档管理平台为主要目标的设计理念:
\begin{itemize}
\item 系统主要的使用者是本校教师和学生。他们是系统主要的信息提供者
\item 以帮助教师和学生提高工作效率为首要目标。系统功能以实际出发帮助他们解决个人文档管理中的实际问题。
\item 由于教师和学生不是专职的信息管理人员,所以要系统必须提供简单易用的使用界面,上手使用必须无需任何额外的学习过程。
\item 系统信息组织不再以集中式的信息收集为主,而是把系统信息以个人文档的形式保存在每个人的私有空间下,每个用户决定这些文档的公开程度和公开范围。
\item 由于系统是面向个人的,所以系统不仅要成为个人文档的存放平台。还要成文个人文档的建立平台。所以系统的功能也更多的偏向于‘纯文本编辑’功能。
\item 如果本系统可以被广大师生接受并大量使用。那么把个人空间内公开的文档整理并展示将不是非常困难的事情,可以说是顺理成章的事情。所以本系统具有老式机构管理库绝大多数功能。
\item 信息管理部门和学校其他职能部门虽然不再是管理信息的主要角色。但是在本系统中他们可以为教师和学生提供类似“信息补全” “成果认证” 等一系列的服务。让用户可以更好的使用系统。
\end{itemize}
通过上面的描述,可以看出本系统的主要设计理念是“面向用户”而非“面向系统”。这一点理念上的变化有点像web 1.0到web 2.0的变化。由门户网站到博客、微博、个人空间的变化。如果用一句话来概括本系统的设计理念的话,应该是“如果用户可以在我们的帮助下把自己的文档和成果管理好,那么我们就可以帮助他们更好的提升并展示自己”。还需要解释的是:由于系统设计理念的不同,所以这套系统在命名上也不再沿用“机构管理库”这样的名词。从功能上主要以管理个人文档和成果为主,所以暂时以“个人学术文档服务平台”命名。

刚才提到系统要以帮助教师和学生解决个人文档管理中遇到的问题为主要目标。那么到底在现在高校中老师和同学们在教学科研和学习中都有那些问题急需解决呢?我们将在下一个小节来汇总一下。

\section{要解决的问题}
\label{sec:question}

上一节提到本系统的设计理念是:帮助每一个用户解决个人文档管理中遇到的问题为主,收集和展示文档资源为辅。现在我们就来分析一下当今高校教师和学生在管理个人文档方面都有哪些问题:
\begin{enumerate}
\item 无论是学术成果的期刊论文、学生的毕业论文、个人编著的书籍、项目的申报和评审材料、技术报告书、教学演示文稿等学术性文档,还是个人的工作笔记、项目日程计划、工作日记等普通文档。在当今高校中,制作这些文档的主要工具是microsoft的office系列软件。office系列软件有“所见即所得”、“易于上手”等诸多优点,在目前社会(起码在中国)上也有相当大的用户群和相当高的普及率,其中以word软件使用频率最高。但是word软件给我们带来的便捷也是有一定局限性的,使用word一段时间以后就会发现它有如下缺点:
  \begin{description}
  \item[难以专心] 写Word文档的时候,我们经常浪费大量时间在Word本身上,特别是那80\%我们用不到的功能。有时看似非常智能的功能,使用起来却有可能成为麻烦的开始。比如项目列表自动添加序号功能,就经常给我们添加一些无用的序号,我们还要费力去删掉它。
  \item[浪费时间在排版上] 使用Word时,我们会花费大量力气去排版,试图让文档变得漂亮一些。是粗体还是斜体,是宋体还是黑体,对创作来说这根本不重要。据了解,一个学生在写作一篇学位论文的过程中用来使用word进行排版格式的时间至少要5个小时。这还是对于一个word操作的熟练的人来说。毫不夸张的说,写毕业论文过程基本就是一个学习word操作的学习过程。
  \item[重复学习成本高] word版本更近快速,从2003,2007,再到2010,每个版本的菜单设置和操作习惯都有所变化,其中以03到07的变化最大,甚至文件格式都不再相同。这样就给用户带来很到的重复学习的成本,好不容易学会来一个版本的操作,下一个版本就出来了,而且操作习惯完全不同。不想升级还不行,因为文件格式的原因,别人用新版本写的文档老版本用户打不开。
  \item[不适合编辑大型文档] word用来排版一些工作报告,宣传册等小型的文档是比较合适的,但是如果用它来编辑学位论文,书籍等大型文档的时候。不仅其所谓的“所见及所得”特性不能完全发挥作用。而且其来带的附加操作会非常的“重”。
  \item[不适合编辑科技文档] word对于编辑以文本格式为主的文档时,可以发挥其特性。但是在编辑科技类文档时,文档的内容就不单纯只用文字了,它会包括数学公式、化学分子式等复杂格式。word在处理这些格式的文本时,其排版的能力就确实一般了。不仅需要安装公式编辑器等插件才可以使用,而且制作好的文档在不同机器上打开时,其样式可能会变得面目全非。
  \item[文档格式封闭] 众所周知,word作为一款闭源软件,用它编辑出来的文档的内部格式是可见的。虽然在07版本之后这种情况有所好转,但是其开放的友好程度依然不好。这样的格式带来了很多问题,首先,封闭格式为二进制文件,而非内容原本,这样就不易做版本控制。其次,封闭格式不易在网络上传播,众所周知,word文档无法在网络上直接查看和编辑,即使复制粘贴成web格式,其格式也会不预期的变化。在这个互联网如此普及的时代,缺失这个特性是非常不方便的。
  \item[价格昂贵] 需要说明的是,这个问题在我国并不是什么真正的问题。首先,word盗版很好获取,我们可以免费的使用这款收费软件。其次,值得称道的是,在我们的高校里,学校已经为我们花钱购买了正版的office系列软件。这样,我们不仅可以免费使用,而且也不用背上“盗贼”的恶名。
  \end{description}
综上所述,word软件的使用给我们的文档编辑带来很多便利,但是对于某些特殊的场合,特别是编写学术性文档的时候,它也未必是我们最佳的选择。总之,无论那种类型的文档,更多的关注文档的内容而非文档的格式和排版,应该是我们创作的最终目标,也是本系统要帮助用户解决的第一个问题。下一个小节中,我们将介绍如何利用\smarkdown和\LaTeX来帮助用户更少的关注排版格式,通过更低的学习成本,更好的完成个人文档的创作。
\item 
\end{enumerate}